\documentclass[10pt,twocolumn,a4]{article}

%% Language and font encodings
\usepackage[italian]{babel}
\usepackage[utf8]{inputenc}
\usepackage[T1]{fontenc}

%% Sets page size and margins
\usepackage[a4paper,top=1cm,bottom=2cm,left=2cm,right=2cm]{geometry}

%% Useful packages
\usepackage{amsmath}
\usepackage{graphicx}

\title{\textbf{Progettazione e realizzazione di un algoritmo di Voice Activity Detection} \\ \vspace{0.5cm} {\large Università degli Studi di Padova}}

\author{Giacomo Camposampiero, matricola 1187180}

\begin{document}
\maketitle

\section{Introduzione}
Gli algoritmi di \textit{Voice Activity Detection} (VAD) sono algoritmi sviluppati allo scopo di rilevare la
presenza di voce umana in tracce audio. Questi algoritmi vengono implementati in una vasta gamma di applicativi
per l'elaborazione del suono e per la comunicazione audio real-time. In questi ultimi in particolare i sistemi
VAD possono rivelarsi molto utili nella riduzione dell'informazione media trasmessa 

Il metodo proposto per l'implementazione di un sistema VAD 



\bibliographystyle{ieee}
\bibliography{references}

\end{document}